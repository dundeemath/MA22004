\documentclass [11pt]{article}
\usepackage {amssymb ,amsmath}
\usepackage {a4}
\usepackage {graphicx}
\usepackage{color}


\topmargin	=-4.mm		% beyond 25.mm
\oddsidemargin	=-3.mm		% beyond 25.mm
\evensidemargin	=-6.mm		% beyond 25.mm
\headheight	=0.mm
\headsep	=0.mm
\textheight	=230mm
\textwidth	=170.mm

\parindent=0pt

\begin{document}



\section*{University of Dundee\hfill Mathematics Division}
\section*{MA22004: Statistics and Probability II : Guide 2019 -- 2020}

\subsection*{Organisation}

The MA22004 module runs for 11 teaching weeks in the second semester,
and is worth 20 SCQF credits (equal to 10 ECTS points).

 All organisation and teaching will be carried out by


\begin{table}[htbp]
\begin{tabular}{ll}
   & {Dr Niall Dodds}\\
 & Mathematics Division\\
 & Room J26,\\
 & Fulton Building\\

\textbf{Tel:}  & 01382 384470\\
\textbf{email:}  & \texttt{ndodds@maths.dundee.ac.uk}
\end{tabular}
\label{tab2}
\end{table}
You should make an appointment to see Dr
Dodds if you have a problem regarding the course.
You may also bring matters of concern about the course 
to the attention of the Mathematics Division Staff/Student Committee, which meets once each semester. 
A volunteer from Level 2 Mathematics will  act as class representative
to sit on the Staff--Student Committee; their name will be posted on BlackBoard.

This module involves 200 hours of student effort, including 66 
contact hours.

\vspace*{-0.25cm}

\subsection*{Timetable} 
The weekly timetable for the module consists of four classes, typically 
1 lecture, 2 workshops and one 2-hour computer laboratory session.

\vspace*{-0.15cm}

\subsection*{Pre-requisite} 

In order to take this course you must have passed module MA12003, or 
an equivalent qualification.




\subsection*{Syllabus}



{\bf Sampling Distributions:}
 Mean and standard deviation of samples, Sampling from 1 poulation, 
Sampling from 2 populations, Shape of sampling distributions. Normal
distribution, Chi-square distribution, F-distribution.

\vspace{0.2in}

{\bf Hypothesis tests:} Null and Alternate hypotheses, Inferences, 
Confidence intervals, Estimating means, proportions and standard deviations.


\vspace{0.2in}

{\bf Linear Regression:} Least squares, Assessing usefulness of a model, Using a model. 


\vspace{0.2in}

{\bf Industrial Quality Control}: Control Charts, Acceptance Sampling.

\vspace{0.2in}

{\bf R software package:} Appropriate use of computational software to carry out statistical and probabalistic calculations.






\subsection*{Assessment}
The overall assessment will be weighted 40\% for Coursework and 60\% 
for a two-hour unseen
Degree Examination.
The Coursework will consist of 2 class tests and 6 computational software assignments.
The assessment weightings are shown in the table below:

\begin{center}
\begin{table}[!h]
\begin{tabular}{ll}
\hline
Class Tests & 20{\%}   \\
Assignments & 20{\%}   \\
Degree Examination & 60{\%}  \\
\hline
\end{tabular}
\label{tab3}
\end{table}
\end{center}

\vspace*{-0.7cm}

%Marks for any Homework not handed in by the deadline given will be deducted at the rate of 
%$10\%$ (per full or part day).  
Assignment deadlines as well as Class Test dates will be posted on MyDundee
and announcements made in the class hours.

\medskip


To pass this module in April/May it is necessary to obtain an overall grade of at least D3 in 
the overall assessment 
\textbf{and} to obtain a grade of at least M1 for the exam {\bf and} to obtain a grade
of at least M1 for the coursework.. 

For those who fail the module in April/May, there may be an opportunity to take a two-hour resit examination paper at the July diet. However unless you have mitigating circumstances then failure to achieve a module grade of CF or above at first attempt may result in you not being permitted to sit the resit exam. Also, unless you have mitigating circumstances, any pass after a resit will be capped at a grade of D3 regardless of the weighted average mark obtained. Resit marks are usually based on the resit exam only. 
\medskip




\subsection*{Your Commitment}

You should attend all classes  except on medical grounds 
or with the special permission of the lecturer concerned.   If you are unable
to attend the degree examination or complete elements of the coursework on time 
then you should inform the Module Leader and submit a medical certificate. Medical certificates should be 
submitted to your School Office as soon as possible after the abscence.

{\bf You must also  submit a Mitigating Circumstances form to explain which aspects of assessment have been affected by your abscence.}

A Medical Certificate will not be taken into account unless 
a Mitigating Circumstances form that refers to the medical certificate has also been completed.

\subsection*{Approved Calculators}

The only types of calculators that have been approved for use in assessments in the School of
Engineering, Physics and Mathematics are the Casio FX83 and the Casio FX85.

\subsection*{Study Support}

If you are  having difficulty with the course you are encouraged to seek help at an early stage 
by making an appointment to see your lecturer.
You may also obtain additional help from the Maths Base (see BlackBoard for details).


\subsection*{End of Module Questionaire}

At the end of each section of the module you will be asked to complete a 
confidential questionnaire regarding the content and presentation of the 
module. This is an important element in the University's Academic Standards 
procedures. 






%\subsection*{Recommended Books}

%TBC





\tiny\begin{flushright}
Last Modified: 14-01--2020
     \end{flushright}





\end{document}
